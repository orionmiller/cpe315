\section{Cache}

Cache - hidden storage for valuables

fast storage for valuable data/instructions that are likely to be accessed
in the future

speeds up access for these data/instructions

cache is essentially a hash table: sotres a key (address), value pair

\subsection{Level 1 Cache}
Commonly used to feed the pipeline.
For instructions - ICache
For Data - DCache

%Diagram of cache bin in pipeline

\subsection{Level 2}
second type of on-chip meory that is larger and slower than l1 chache

typically l2 cache is unified, holds both instruction and data

if instruction or data i not found in l1, hopefully it will be in l2

%l2 cache diagram

\subsection{Level N Cache \& Beyond}
accessing off-chip memory very slow because:
- large capacitance from macroscopic pins on cpu package
- large capacitance from circuit board wiring connecting CPU and off-chip memory

We want more on-chip memory

%diagram if flow form pipeline - l1 - l2 - main memory

-main memory acces time
AMD Athlon FX-51 memory access is 125 clock cycles = 57 ns
intel 3.2GHz P4 memory access is 204 clock cycles = 64 ns

\subsection{Principle of Locality}
Programs access a small proportion of their address space at any time.
temporal locality
- items accessed recently are likely to be access again soon
e.g. instructions in a loop, induction variables

Spatial Locality
- items near those accessed recently are likely to be accessed soon
e.g. sequential instruction access, array data

Take Advantage of this
- memory hierarchy
- store everything on disk
- copy recently access (and nearby) items
  from disk to smaller DRAM memory
  - main memory
- copy more recently accessed (and nearby) items
from DRAM to smaller SRAM memory
  - cache memory attached to cpu

\subsection{Memory Hierarchy Levels}
- block (aka line): unit of copying
- may be multiple words
- if accessed data is present in upper level
  - hit: access satisfied by upper level
  - hit ratio: hits/accesses
if accessed data is absent
  - miss: block copied from lower level
    - time take: miss penalty
    - miss ratio: misses/accesses
      = 1 - hit ratio
    - then accessed data supplied from upper level

\subsection{Direct Mapped}
where in cache should data be placed so it can be quickly located?
cache size is selecte to be 2 to the power of i e.g.
the entire 32-bit address is used to locate an instruction/data in main memory
the lower i bits (e.g. 16) are used to to (quickly) find an item in the cache
implies that each instruction/data from main memory is copied to (mapped to) a specific
cache location

\subsection{Set Associative}

\subsection{Performance}
